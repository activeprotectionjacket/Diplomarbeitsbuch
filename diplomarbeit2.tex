\section{Kapitel aus der anderen
Datei}\label{kapitel-aus-der-anderen-datei}

Dieses Kapitel wurde als \emph{diplomarbeit2.md} geschrieben und dann
mit \emph{pandoc} in \TeX~umgewandelt.

\begin{lstlisting}[language=bash]
pandoc --listings -s diplomarbeit2.md -o diplomarbeit2.tex 
\end{lstlisting}

Wie man sieht ist das ganz einfach, sogar Listings sind möglich. Weitere
Optionen sind möglich bzw. sinnvoll -- siehe Kapitel \ref{skripts},
Seite \pageref{skripts}. Man beachte die von Pandoc automatisch
genmerierten Label für Querverweise.

Vorschlag zur Durchführung:

\begin{itemize}
\tightlist
\item
  ein Ordner mit den Pandoc-Dateien
\item
  ein Skript/Batch-Datei erzeugt daraus die Latex-Dateien
\item
  in einem neuen Ordner -- das erhöht die Übersichtlichkeit
\item
  die Latex-Dateien werden dann in das Hauptdokument eingebunden
\end{itemize}

Und nun zu einem Bild.

\begin{figure}[htbp]
\centering
\includegraphics{HTL3RLogo.png}
\caption{Der Text steht unterhalb}
\end{figure}

Achtung: Pandoc skaliert die Bilder nicht! Hier hilft nur eine
vorhergehende Skalierung des Bildes. Oder nachträgliches Editieren --
ganz einfach die passende Breite in der \emph{.tex} Datei ausbessern.

Man kann auch die Breite aber auch durch La\TeX~Befehle angeben -- das
ändert aber die Standardbreite aller folgenden Bilder!

\setkeys{Gin}{width=0.6\textwidth,}

\begin{figure}[htbp]
\centering
\includegraphics{HTL3RLogo.png}
\caption{Das kleinere Bild}
\end{figure}

\setkeys{Gin}{width=2cm}

\begin{figure}[htbp]
\centering
\includegraphics{HTL3RLogo.png}
\caption{Das ganz kleine Bild}
\end{figure}

Auch Listen sind kein Problem, wichtig sind nur Leerzeilen zwischen den
Listenpunkten. Hier sieht man eine einfache Aufzählung.

\begin{itemize}
\item
  wichtig
\item
  auch ganz lange Texte können bei Listen geschrieben werden.

  Sogar mehrere Absätze sind möglich.
\item
  Ende der Liste.
\end{itemize}

Welches Zeichen am Anfang der Liste steht ist dabei leicht einzustellen,
im \emph{pandoc} Manual gibt es nähere Infos:

\begin{enumerate}
\def\labelenumi{\arabic{enumi}.}
\item
  eins
\item
  zwei

  \begin{enumerate}
  \def\labelenumii{\roman{enumii}.}
  \tightlist
  \item
    zwei eins -- Mindestens 4 Zeichen eingerückt
  \item
    zwei zwei
  \end{enumerate}
\item
  drei. \emph{Pandoc} zählt richtig, das Zeichen am Anfang der Zeile ist
  nur ein Muster!
\end{enumerate}

Mit den richtigen Optionen werden URLs automatisch richtig dargestellt
und sind im fertigen Pdf auch \emph{klickbar}:
\url{http://pandoc.org/README.html}
