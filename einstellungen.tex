\section{Was man einstellen kann}\label{was-man-einstellen-kann}

Einige \emph{Dinge} kann man bei der Diplomarbeit auch noch anpassen.
Einiges ist sogar in der lyx- bzw. tex-Datei schon vorgesehen -- bitte
den Anfang des Source-Codes aufmerksam lesen.

\subsection{Inhalt/Kapitel}\label{inhaltkapitel}

Einige Ideen zur Gliederung der Arbeit gibt es unter
\url{http://www.diplomarbeiten-bbs.at/} \footnote{\url{http://www.diplomarbeiten-bbs.at/erstellung/durchf\%C3\%BChrung/gliederung-der-diplomarbeit-und-formale-vorgaben-0}}

\subsection{Einseitig/zweiseitig}\label{einseitigzweiseitig}

Korrekturexemplar immer einseitig drucken. Beim fertigen Buch sind beide
Varianten möglich -- bitte mit dem Betreuer abklären.

\subsection{Farben}\label{farben}

\begin{itemize}
\tightlist
\item
  Deckblatt mit Logo immer in Farbe
\item
  Farbausdrucke sind viel teurer -- Notwendigkeit prüfen und mit dem
  Betreuer abklären.
\item
  am aufwändigsten: ein paar Seiten in Farbe
\item
  falls Schwarz-Weiß: kein Logo beim laufenden Text
\end{itemize}

\subsection{Autor}\label{autor}

Bei der Diplomarbeit muss jeder Text einem Autor zuzuordnen sein. Man
kann das wie in der Vorlage in der Fußzeile machen. Alternativ kann man
auch eine Übersicht als Anhang eingefügt werden -- bitte mit dem
Betreuer abklären.

\subsection{Absätze}\label{absuxe4tze}

\subsubsection{Einzug}\label{einzug}

Den Beginn eines neuen Absatzes kann man durch Abstand oder durch
Einrücken kennzeichnen.

Diese Einstellung wird in der Vorlage bzw. im Header gemacht:

\begin{lstlisting}
%\parindent0pt % auskommentieren, wenn keine Einrueckung der 
               % ersten Absatzzeile gewuenscht
%\parskip1.5ex plus0.5ex minus0.5ex % flexibler Absatzabstand
\end{lstlisting}

Die Option \lstinline!parskip=half! bei \lstinline!documentclass!
ersetzt bereits den Absatzeinzug durch einen Absatzabstand.

oder \url{http://ctan.org/pkg/parskip}

\begin{lstlisting}
\usepackage[parfill]{parskip}
\end{lstlisting}

\subsubsection{noch schöner}\label{noch-schuxf6ner}

\url{http://www.khirevich.com/latex/microtype/}

\subsection{Aufzählungen}\label{aufzuxe4hlungen}

Man kann die Einrückung und vieles mehr anpassen:

\begin{lstlisting}
\usepackage{enumitem}
\setlist[1]{labelindent=\parindent}
\setlist{align=left}
\setlist[itemize]{leftmargin=*}

% oder: 
\setlength\partopsep{0.5ex}
\end{lstlisting}

\subsection{Warnungen}\label{warnungen}

\begin{lstlisting}
%\sloppy  % etwas laxere Abstandskontrolle (weniger Fehlermeldungen)
\end{lstlisting}

\subsection{Listings}\label{listings}

Sonderzeichen:

\url{http://en.wikibooks.org/wiki/LaTeX/Source_Code_Listings\#Encoding_issue}

Anpassen:

\url{http://stackoverflow.com/questions/1965702/how-to-mark-line-breaking-of-long-lines}

\url{http://www.bollchen.de/blog/2011/04/good-looking-line-breaks-with-the-listings-package/}

\subsection{Zitierstil:}\label{zitierstil}

Bitte mit dem Betreuer abklären.

\begin{itemize}
\tightlist
\item
  plaindin - mit nummern {[}1{]}
\item
  alphadin - mit Text+Jahr {[}Hor99{]}
\end{itemize}

\subsection{Glossar}\label{glossar}

\begin{itemize}
\tightlist
\item
  \url{http://texwelt.de/wissen/fragen/10496/glossaries-alle-symbole-nur-verwendete-abkurzungen-anzeigen}
\item
  \url{http://en.wikibooks.org/wiki/LaTeX/Glossary} für die
  verschiedenen Formen
\end{itemize}

\subsection{Ausdruck zu weit
oben/unten}\label{ausdruck-zu-weit-obenunten}

Man kann das Layout anpassen: Position auf Seite

\begin{lstlisting}
\voffset10mm
\end{lstlisting}

\subsection{URLs}\label{urls}

Man kann/sollte das \lstinline!hyperref!-Paket anpassen, die bunten
Links kann man ausschalten.

\begin{lstlisting}
\hypersetup{breaklinks=true,
bookmarks=true,
pdfauthor={Mein Name},% <------------------- anpassen!
pdftitle={Die Diplomarbeit},% <------------------- anpassen!
colorlinks=true,
citecolor=blue,
urlcolor=blue,
linkcolor=magenta,
pdfborder={0 0 0}}

\urlstyle{same}
\end{lstlisting}

\subsection{Pandoc-Caption mit
Verweis}\label{pandoc-caption-mit-verweis}

siehe \url{https://github.com/chiakaivalya/thesis-markdown-pandoc}

This is how you insert figures using markdown. Also how to insert
citations copied over from your bibliography manager (I specifically
used Pandoc Citations in Papers).

\begin{lstlisting}
![Figure from Walczak, 2010[@Walczak:2010uk]. \label{mitosis} ](figures/mitosis_Walczak.pdf)
\end{lstlisting}

\subsection{Seitennummern}\label{seitennummern}

\url{http://www.golatex.de/wiki/\%5Cfrontmatter}

\begin{lstlisting}
\frontmatter % switches to roman numbering
\mainmatter
\backmatter
\end{lstlisting}

\subsection{TU Wien -- Informatik}\label{tu-wien-informatik}

\begin{itemize}
\tightlist
\item
  \url{https://gitlab.cg.tuwien.ac.at/auzinger/vutinfth.git}

  \begin{itemize}
  \tightlist
  \item
    super Vorlage und Build-Skripts (auch für Windows)
  \end{itemize}
\item
  \url{http://www.informatik.tuwien.ac.at/dekanat/abschluss-master}
\item
  \url{http://www.informatik.tuwien.ac.at/fakultaet/informatik-code-of-ethics.pdf}
\item
  Alte Vorlage

  \begin{itemize}
  \tightlist
  \item
    \url{http://ieg.ifs.tuwien.ac.at/~aigner/download/tuwien.sty}
  \item
    von dort sind die Abkürzungen kopiert
  \end{itemize}
\end{itemize}

\section{Skripts}\label{skripts}

\subsection{Pandoc nach Latex}\label{pandoc-nach-latex}

\begin{lstlisting}[language=bash]
#! /bin/bash
#set -x
#set -v
set -e

PANDOCMODULES=markdown+auto_identifiers
PANDOCMODULES=${PANDOCMODULES}+definition_lists
#PANDOCMODULES=${PANDOCMODULES}+compact_definition_lists
PANDOCMODULES=${PANDOCMODULES}+fenced_code_attributes
PANDOCMODULES=${PANDOCMODULES}+autolink_bare_uris
PANDOCMODULES=${PANDOCMODULES}+simple_tables+table_captions
PANDOCMODULES=${PANDOCMODULES}+inline_notes+footnotes


PANDOCOPT="--listings-S -N -f ${PANDOCMODULES}"


mkdir -p ../kaptex/
rm -f ../kaptex/*

for f in *.md
do
   out=$(basename $f .md).tex
   echo -n $f " "
   pandoc ${PANDOCOPT} $f -o ../kaptex/$out
done
\end{lstlisting}

\subsection{Diplomarbeit bauen}\label{diplomarbeit-bauen}

Wichtig -- damit alle Seitenummern und Verweise passen:

\begin{itemize}
\tightlist
\item
  erster Latex-Lauf
\item
  \lstinline!makeindex! und \lstinline!bibref! aufrufen
\item
  noch zweimal Latex
\end{itemize}

\begin{lstlisting}[language=bash]
cd kapmd
./create.sh
cd ..
pdflatex diplbuch.tex &&
makeindex -c -q diplbuch.idx &&
bibtex diplbuch
pdflatex diplbuch.tex &&
pdflatex diplbuch.tex
\end{lstlisting}

